\chapter{Conclusion and Future Work}
In this work we have introduced Simulated Annealing as an alternative estimation procedure for the Markov Blanket.
While the Annealing MAP did not result in performance competitive to the GLASSO, it was shown to be comparable to the previous approach by \cite{kaufmann_bayesian_2015}.
In addition, SA has proven to be more robust in its application on the HIV-X data set by allowing for a smoother process of network selection, while the BMB gave unexpected problems.
The feasible $\lambda$ region of the BMB lies closely to the region subject to convergence problems of the Markov chains, thus making the inferred networks less certain.
In the test data however, SA had the disadvantage of an upper limit on the sparsity, presumably due to numerical instability.
We assume that this instability mainly stems from the MGIG distributed $\Wxx$ posterior conditional. 

In applications of both models, we could identify multi-modality of the posterior marginal of $\Wxy$ for small $\lambda$, while posterior marginals for higher $\lambda$ seem to be unimodal.
This is also reflected in the Annealing behavior, as the sampler starts fluctuating between multiple modes when in the corresponding $\lambda$ region.

Finally, the application on the HIV-X dataset indicated that the GRT mutations may not have a significant effect on the clinical factors in comparison to the haplotype mutations. However, it is unclear whether the seemingly non-normal distributions of the latent values for the GRT mutations influenced this result.
 Although we could identify some connections in the networks that are agreeable with known results, a lot of edges express counter-intuitive partial correlations.
For example, multiple resistance relevant mutations are negatively correlated with the ddPCR HIV cell count.


\section*{Future Work}
One of the main shortcomings of the model underlying the BMB is the reliance on the MGIG distribution.
In general, the MGIG proves to be difficult both in theory and practice due to the suboptimal methods available for sampling from it.
Thus future work could be focused on finding an different sampling method.
A possible alternative for directly sampling the MGIG would be substituting the draw of the MGIG with a draw from the $\Wxx$ marginal $P(\Wxx|\matr{T}, \matr{S}, \lambda)$.
While the marginal does not offer a closed form solution in terms of a known distribution, we can use a Metropolis Hastings sampler.
Due to the similarity, we would suggest the Wishart arising from the special case of $P(\Wxx|\matr{T}, \matr{S}, \lambda)$ for $\matr{D}\rightarrow\matr{0}$ (see \autoref{ss:postmarginal}) as proposal function.
The difference between the special case and the general case of the marginal posterior lies in the determinant:
$$\det
\Big(
(\matr{S}_{22} + \matr{I}) \otimes \matr{W}_{11}^{-1} +\matr{D}
\Big)$$
We know that $A=((\matr{S}_{22} + \matr{I}) \otimes \matr{W}_{11}^{-1})$ is symmetric and positive definite (so all eigenvalues of $A$ are positive), and $\matr{D}$ is a diagonal matrix with positive entries.
With $\det(A)=\prod_{i}\lambda_i$ and $tr(A) = \sum_i \lambda_i$ it's clear to see
that
$$\det\Big((\matr{S}_{22} + \matr{I}) \otimes \matr{W}_{11}^{-1} + \matr{D}\Big) > \det\Big((\matr{S}_{22} + \matr{I}) \otimes \matr{W}_{11}^{-1}\Big)
$$
As only the normalization differs, we assume that the general case of the posterior marginal is a flatter version of the special case.
Additionally, $\matr{D}$ tends to be quite small in practice, so the special case should serve as a good proposal function with a high acceptance rate.
\\\\
Another interesting point would be adjusting the synthetic data to a more general setting.
We have seen that the usability of the Annealing approach played a big role for the application on real data.
However, the cause of this difference compared to the BMB is unclear, as the tests on artificial data did not indicate similar problems.
While the artificial data does reflect the small world property of real problems, the already normally distributed data presumably obsoletes the semi-parametric copula.
In contrast, many realistic use cases with mixed data rely on the copula transformation,
which should motivate the use of a more appropriate data set.
Because of this, we would suggest comparing the BMB and the Annealing on unbalanced and sparse data (similar to the mutations) that still exhibits small world properties.
